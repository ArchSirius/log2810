\documentclass[10pt,letterpaper]{article}

%Langue et caractères spéciaux
\usepackage[french]{babel} 
\usepackage[T1]{fontenc}
\usepackage{lmodern}
\usepackage[utf8]{inputenc}
\usepackage{textcomp}
\usepackage{lipsum}
%Marges
\usepackage[top=2cm, bottom=2cm, left=2cm, right=2cm, columnsep=20pt]{geometry}
\usepackage{scrextend}
%Math
\usepackage{amsmath}
%Graphiques
\usepackage{graphicx}

%%%%%%%%%%%%
% Document %
%%%%%%%%%%%%
\begin{document}

%%%%%%%%%%%%%%
% Page titre %
%%%%%%%%%%%%%%
\begin{titlepage}
\begin{center}

\includegraphics[width=0.25\textwidth]{./logo.png}~\\[1cm]

\textsc{\huge École Polytechnique de Montréal}\\[1.5cm]

\rule{0.5\linewidth}{0.5mm} \\[0.4cm]
{\LARGE TP1}\\[0.4cm]
{\Large Graphes}\\[1.0cm]

{\large LOG2810}\\[0.4cm]
{\large \textbf{Structures discrètes}}\\[0.4cm]

\rule{0.5\linewidth}{0.5mm} \\[1.0cm]

{\large par}\\[0.6cm]
\begin{Large}
  \begin{tabular}{r l}
    Jules \textsc{Favreau-Pollender} & 1718752\\[0.4cm]
    Francis \textsc{Rochon} & 1721338\\[0.4cm]
    Samuel \textsc{Rondeau} & 1723869\\[0.4cm]
  \end{tabular}
\end{Large}


\vfill

{\large Remis le}\\[0.3cm]
{\Large \today}\\[1.5cm]
{\large Département de génie informatique et logiciel}\\[0.3cm]
{\large École Polytechnique de Montréal}

\end{center}
\end{titlepage}



\section{Introduction}
\hrule
\vspace{1em}
Dans le cadre de notre future carrière en informatique, il peut parfois être difficile de modéliser 
certaines applications réelles en applications informatiques si on ne connaît pas les outils nécessaires. 
Parmi ces outils, nous avons vu des notions sur la théorie des graphes qui permettent de simuler plusieurs situations du monde réel.\\
\\
Ainsi, durant ce premier travail pratique, nous aurons l’occasion de modéliser un réseau informatique local. 
Un réseau informatique est un regroupement d’appareils informatiques connectés entres eux, 
par fils ou par connexion sans fil, dans le but d’échanger de l’information. 
En fait, on peut représenter chacun des appareils par des sommets d’un graphe (ou des nœuds) dont chacun possède des caractéristiques et des contraintes de connexion. 
Nous pouvons ainsi les relier par des arrêtes pour représenter une connexion entre certains appareils selon leur compatibilité. 
Nous pouvons même allez jusqu’à mettre des chiffres sur chacune des connexions pour représenter le coût d’une connexion 
et ainsi calculer le coût total de l’échange de l’information d’un point à un autre en additionnant chacun des coûts associés aux connexions nécessaires. 
Suivant ce principe, nous pouvons déterminer quel réseau de connexions entre deux appareils est le plus économique, 
dans la cas où plus d’une façon de communiquer est possible.\\
\\
Nous pouvons donc représenter ce système par un graphe dont les sommets sont les appareils et les arcs sont les connexions entre ces appareils, 
et y appliquer l’algorithme de Floyd-Warshall pour calculer le plus faible coût entre deux sommets. 
On réussit donc à modéliser ce réseau informatique complexe en quelque chose de simple et de concret.\\
\\
Dans ce rapport, nous présenterons une explication de la solution adoptée pour réaliser ce travail et nous aborderons les différentes difficultés rencontrées.



\newpage
\section{Présentation du travail}
\hrule
\vspace{1em}
Notre première action a été de bien lire et relire l’énoncé puis de s’assurer que nous partagions tous la même vision du projet. 
Nous avons opté pour une conception orientée objet où chaque appareil est une classe dérivée d’une classe abstraite Noeud.\\
\\
Le réseau aussi est une classe, qui contient un conteneur de type Map de noeud, avec leurs identifiants comme clés. 
Ce procédé nous assure qu’il n’y a aucun doublon dans la réseau. 
Ce conteneur ne contient pas les informations de connexion. 
Il ne contient que les appareils lus. 
Ce sont les noeuds eux-mêmes qui possèdent un vecteur vers d’autres noeuds, symbolisant les connexions. 
Cependant, le conteneur Map n’était pas propice à une bonne implémentation de l’algorithme de Floyd-Warshall. 
Nous avons donc également créé une matrice d’adjacence, constituée d’un vecteur de vecteurs d’entiers symbolisant les coûts de connexion. 
L’utilisation de vecteurs est nécessaire car un simple tableau de taille inconnue non constante à la compilation n’est pas valide. 
Une première matrice détermine le coût de chaque connexion entre deux appareils, 
et une seconde matrice résultante suite à l’algorithme de Floyd-Marshall détermine le coût total le plus faible entre toute paire possiblement reliée.\\
\\
Par l’entremise de booléens, on s’assure que l’algorithme n’est exécuté que lorsque nécessaire. 
Par exemple, si on demande de vérifier deux coûts, la matrice de coûts totaux n’est générée qu’une fois, jusqu’à la prochaine modification nécessaire.\\
\\
Par l’entremise du conteneur de noeuds de la classe réseau, on peut facilement afficher la topologie du réseau entier, 
puisque chaque noeud possède sa propre fonction d’affichage. 
On peut également effectuer les opérations demandées, telles qu’afficher le nombre de ports disponibles d’un commutateur ou d’un routeur.\\
\\
L’ajout d’un noeud au réseau se fait de deux façons. 
On peut simplement ajouter un nouvel appareil au réseau, sans préciser de connexion. 
On peut également ajouter l’appareil en spécifiant qu’il se connecte à un appareil déjà présent dans le réseau. 
Dans ce dernier cas, les règles de connexion s’appliquent : vérification du type de connexion, nombre de ports disponibles, etc.\\
\\
Remplacer un noeud est un peu plus problématique. 
Le remplacement se produit en deux étapes : d’abord on retire le noeud cible en le déconnectant et en le détruisant, 
ensuite on ajoute le nouveau noeud en le connectant aux appareils qui étaient connectés à l’ancien. 
La suppression d’un noeud nous aura longtemps posé problème, généralement par des erreurs de segmentation. 
La gestion de la mémoire fut plus difficile qu’anticipé. 
Cependant, toutes les vérifications nécessaires sont effectuées avec succès.\\
\\
Somme toutes, notre implémentation du réseau est logique et l’algorithme de Floyd-Marshall fonctionne à merveille. 
Bien qu’elle ait apporté quelques soucis de gestion de la mémoire et de questionnements quant à l’utilisation de pointeurs, 
nous croyons que notre conception est logique et fiable.



\newpage
\section{Difficultés rencontrées}
\hrule
\vspace{1em}
Durant la réalisation de ce projet, nous avons rencontré quelques difficultés. 
L’une d’entre elles était le problème de compatibilité du code source entre chacun de nous. 
Étant donné que l’un travaille sous Linux et les deux autres sur Windows avec Microsoft Visual Studio 2013, 
nous avons eu quelques incompatibilités lors de la compilation. 
Certaines méthodes étaient acceptées sous Linux, mais elles étaient refusées sur Visual Studio, et vice-versa. 
Par exemple, << strncpy >> qui devait être remplacé par << strncpy$\_$s >> sous Windows à chaque fois que l’on transmettait notre code.\\
\\
Une des plus grandes difficultés rencontrées était que nous avons dû faire la majeure partie du travail pendant la semaine de relâche, 
en plus d'un autre important projet pour un autre cours. 
Ainsi, certains d’entre nous ont voyagé chez de la famille éloignée, ce qui rendait impossible les rencontres en personne. 
Par conséquent, cela rendait la coordination plus difficile. 
Pour pallier à cela, nous nous sommes entendus pour utiliser une messagerie instantanée comme moyen de communication 
et nous avons utilisé le logiciel de gestion de versions Git pour partager notre code source. 
Cette façon de fonctionner est une bonne idée, mais n’est pas sans difficultés. 
L’utilisation de Git était idéale pour la tâche à faire, seul un membre de notre équipe connaissait bien Git; les deux autres devant être initiés pour ce travail. 
Nous avons ainsi vécu quelques problèmes au début, mais nous nous sommes adaptés rapidement. 
Néanmoins, Git nous a posé d’autres complications. 
Étant donné que nous travaillions chacun de notre côté sans se voir en personne, il est arrivé que nous travaillions tous les trois sur le même fichier. 
De la sorte que, lors de notre envoi du code vers le serveur distant Git, beaucoup de conflits de code sont survenus, 
puisque nous modifions des parties de code entrant en conflit avec les modifications des autres. 
Une des solutions adoptées pour réduire ce désagrément a été de planifier des rencontres avec une 
messagerie instantanée vocale pour communiquer agréablement et en temps réel. 
Skype était idéal dans notre cas, car il nous a permis d’avoir une conversation à trois personnes et même de 
saisir du texte lorsque nous voulions partager des ressources ou morceaux de code rapidement.\\
\\
De plus, une difficulté non négligeable de ce TP était de revenir au C++. 
En effet, nous étions un peu << rouillés >> avec ce langage, puisque les autres travaux de cette session sont programmés en Java. 
Heureusement, nous avons su nous réadapter rapidement.\\
\\
Nous avons également eu des difficultés à bien comprendre l’algorithme de Floyd-Warshall. 
Nous nous sommes posé comme question s’il était mieux d’implémenter une matrice de nœuds 
ou si nous pouvions utiliser directement notre conteneur de type << Map >> déjà présent dans la classe réseau. 
Nous avons finalement choisit d’opter pour les deux. Un simple matrice donnant les coûts, 
et toutes les connexions entre appareils avec et sans fil aurait était complexe à implémenter. 
Un Map efficace ne contient qu’une clé et un noeud, et nous voulions garder ce format. 
L’utilisation des deux solutions permet d’avoir une solution complète en deux parties simples, plutôt qu’une seule complexe.\\
\\
Finalement, nous avons hésité à savoir si l’on devait utiliser des pointeurs pour remplir notre conteneur de nœuds, 
des références ou des pointeurs intelligents. 
Nous avons fait le choix de conserver notre première implémentation, soit de créer les objets dans la boucle d’exécution principale, 
mais de n’utiliser que des pointeurs dans tout le reste du programme. 
Les copies par valeurs et les références causaient plus de problèmes que l’utilisation de pointeurs. 
Nous avons rejeté les pointeurs intelligents puisqu’ils sont complexes à gérer. 
Un simple pointeur sans gestion automatique de la mémoire est facile d’utilisation, mais nous avons vécu des problèmes de gestion de la mémoire plus tard.



\newpage
\section{Conclusion}
\hrule
\vspace{1em}
\lipsum[4]



\vspace{2em}
\section{Critiques et améliorations}
\hrule
\vspace{1em}
\lipsum[5]

\end{document}