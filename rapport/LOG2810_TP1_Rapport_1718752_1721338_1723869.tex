\documentclass[10pt,letterpaper]{article}

%Langue et caractères spéciaux
\usepackage[french]{babel} 
\usepackage[T1]{fontenc}
\usepackage{lmodern}
\usepackage[utf8]{inputenc}
\usepackage{textcomp}
\usepackage{lipsum}
%Marges
\usepackage[top=2cm, bottom=2cm, left=2cm, right=2cm, columnsep=20pt]{geometry}
\usepackage{scrextend}
%Math
\usepackage{amsmath}
%Graphiques
\usepackage{graphicx}

%%%%%%%%%%%%
% Document %
%%%%%%%%%%%%
\begin{document}

%%%%%%%%%%%%%%
% Page titre %
%%%%%%%%%%%%%%
\begin{titlepage}
\begin{center}

\includegraphics[width=0.25\textwidth]{./logo.png}~\\[1cm]

\textsc{\huge École Polytechnique de Montréal}\\[1.5cm]

\rule{0.5\linewidth}{0.5mm} \\[0.4cm]
{\LARGE TP1}\\[0.4cm]
{\Large Graphes}\\[1.0cm]

{\large LOG2810}\\[0.4cm]
{\large \textbf{Structures discrètes}}\\[0.4cm]

\rule{0.5\linewidth}{0.5mm} \\[1.0cm]

{\large par}\\[0.6cm]
\begin{Large}
  \begin{tabular}{r l}
    Jules \textsc{Favreau-Pollender} & 1718752\\[0.4cm]
    Francis \textsc{Rochon} & 1721338\\[0.4cm]
    Samuel \textsc{Rondeau} & 1723869\\[0.4cm]
  \end{tabular}
\end{Large}


\vfill

{\large Remis le}\\[0.3cm]
{\Large \today}\\[1.5cm]
{\large Département de génie informatique et logiciel}\\[0.3cm]
{\large École Polytechnique de Montréal}

\end{center}
\end{titlepage}



\section{Introduction}
\hrule
\vspace{1em}
Dans le cadre de notre future carrière en informatique, il peut parfois être difficile de modéliser 
certaines applications réelles en applications informatiques si on ne connaît pas les outils nécessaires. 
Parmi ces outils, nous avons vu des notions sur la théorie des graphes qui permettent de simuler plusieurs situations du monde réel.\\
\\
Ainsi, durant ce premier travail pratique, nous aurons l’occasion de modéliser un réseau informatique local. 
Un réseau informatique est un regroupement d’appareils informatiques connectés entres eux, 
par fils ou par connexion sans fil, dans le but d’échanger de l’information. 
En fait, on peut représenter chacun des appareils par des sommets d’un graphe (ou des nœuds) dont chacun possède des caractéristiques et des contraintes de connexion. 
Nous pouvons ainsi les relier par des arrêtes pour représenter une connexion entre certains appareils selon leur compatibilité. 
Nous pouvons même allez jusqu’à mettre des chiffres sur chacune des connexions pour représenter le coût d’une connexion 
et ainsi calculer le coût total de l’échange de l’information d’un point à un autre en additionnant chacun des coûts associés aux connexions nécessaires. 
Suivant ce principe, nous pouvons déterminer quel réseau de connexions entre deux appareils est le plus économique, 
dans la cas où plus d’une façon de communiquer est possible.\\
\\
Nous pouvons donc représenter ce système par un graphe dont les sommets sont les appareils et les arcs sont les connexions entre ces appareils, 
et y appliquer l’algorithme de Floyd-Warshall pour calculer le plus faible coût entre deux sommets. 
On réussit donc à modéliser ce réseau informatique complexe en quelque chose de simple et de concret.\\
\\
Dans ce rapport, nous présenterons une explication de la solution adoptée pour réaliser ce travail et nous aborderons les différentes difficultés rencontrées.



\newpage
\section{Présentation du travail}
\hrule
\vspace{1em}
\lipsum[1]~\\
\lipsum[2]~\\
\lipsum[3]



\newpage
\section{Difficultés rencontrées}
\hrule
\vspace{1em}
Durant la réalisation de ce projet, nous avons rencontré quelques difficultés. 
L’une d’entre elles était le problème de compatibilité du code source entre chacun de nous. 
Étant donné que l’un travaille sous Linux et les deux autres sur Windows avec Microsoft Visual Studio 2013, 
nous avons eu quelques incompatibilités lors de la compilation. 
Certaines méthodes étaient acceptées sous Linux, mais elles étaient refusées sur Visual Studio, et vice-versa. 
Par exemple, << strncpy >> qui devait être remplacé par << strncpy$\_$s >> sous Windows à chaque fois que l’on transmettait notre code.\\
\\
Une des plus grandes difficultés rencontrées était que nous avons dû faire la majeure partie du travail pendant la semaine de relâche, 
en plus d'un autre important projet pour un autre cours. 
Ainsi, certains d’entre nous ont voyagé chez de la famille éloignée, ce qui rendait impossible les rencontres en personne. 
Par conséquent, cela rendait la coordination plus difficile. 
Pour pallier à cela, nous nous sommes entendus pour utiliser une messagerie instantanée comme moyen de communication 
et nous avons utilisé le logiciel de gestion de versions Git pour partager notre code source. 
Cette façon de fonctionner est une bonne idée, mais n’est pas sans difficultés. 
L’utilisation de Git était idéale pour la tâche à faire, seul un membre de notre équipe connaissait bien Git; les deux autres devant être initiés pour ce travail. 
Nous avons ainsi vécu quelques problèmes au début, mais nous nous sommes adaptés rapidement. 
Néanmoins, Git nous a posé d’autres complications. 
Étant donné que nous travaillions chacun de notre côté sans se voir en personne, il est arrivé que nous travaillions tous les trois sur le même fichier. 
De la sorte que, lors de notre envoi du code vers le serveur distant Git, beaucoup de conflits de code sont survenus, 
puisque nous modifions des parties de code entrant en conflit avec les modifications des autres. 
Une des solutions adoptées pour réduire ce désagrément a été de planifier des rencontres avec une 
messagerie instantanée vocale pour communiquer agréablement et en temps réel. 
Skype était idéal dans notre cas, car il nous a permis d’avoir une conversation à trois personnes et même de 
saisir du texte lorsque nous voulions partager des ressources ou morceaux de code rapidement.\\
\\
De plus, une difficulté non négligeable de ce TP était de revenir au C++. 
En effet, nous étions un peu << rouillés >> avec ce langage, puisque les autres travaux de cette session sont programmés en Java. 
Heureusement, nous avons su nous réadapter rapidement.



\newpage
\section{Conclusion}
\hrule
\vspace{1em}
\lipsum[4]



\vspace{2em}
\section{Critiques et améliorations}
\hrule
\vspace{1em}
\lipsum[5]

\end{document}